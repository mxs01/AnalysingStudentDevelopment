%%%%%%%% ICML 2023 EXAMPLE LATEX SUBMISSION FILE %%%%%%%%%%%%%%%%%

\documentclass{article}

% Recommended, but optional, packages for figures and better typesetting:
\usepackage{microtype}
\usepackage{graphicx}
\usepackage{subfigure}
\usepackage{booktabs} % for professional tables

\usepackage{tikz}
% Corporate Design of the University of Tübingen
% Primary Colors
\definecolor{TUred}{RGB}{165,30,55}
\definecolor{TUgold}{RGB}{180,160,105}
\definecolor{TUdark}{RGB}{50,65,75}
\definecolor{TUgray}{RGB}{175,179,183}

% Secondary Colors
\definecolor{TUdarkblue}{RGB}{65,90,140}
\definecolor{TUblue}{RGB}{0,105,170}
\definecolor{TUlightblue}{RGB}{80,170,200}
\definecolor{TUlightgreen}{RGB}{130,185,160}
\definecolor{TUgreen}{RGB}{125,165,75}
\definecolor{TUdarkgreen}{RGB}{50,110,30}
\definecolor{TUocre}{RGB}{200,80,60}
\definecolor{TUviolet}{RGB}{175,110,150}
\definecolor{TUmauve}{RGB}{180,160,150}
\definecolor{TUbeige}{RGB}{215,180,105}
\definecolor{TUorange}{RGB}{210,150,0}
\definecolor{TUbrown}{RGB}{145,105,70}

% hyperref makes hyperlinks in the resulting PDF.
% If your build breaks (sometimes temporarily if a hyperlink spans a page)
% please comment out the following usepackage line and replace
% \usepackage{icml2023} with \usepackage[nohyperref]{icml2023} above.
\usepackage{hyperref}


% Attempt to make hyperref and algorithmic work together better:
\newcommand{\theHalgorithm}{\arabic{algorithm}}

\usepackage[accepted]{icml2023}

% For theorems and such
\usepackage{amsmath}
\usepackage{amssymb}
\usepackage{mathtools}
\usepackage{amsthm}

% if you use cleveref..
\usepackage[capitalize,noabbrev]{cleveref}

%%%%%%%%%%%%%%%%%%%%%%%%%%%%%%%%
% THEOREMS
%%%%%%%%%%%%%%%%%%%%%%%%%%%%%%%%
\theoremstyle{plain}
\newtheorem{theorem}{Theorem}[section]
\newtheorem{proposition}[theorem]{Proposition}
\newtheorem{lemma}[theorem]{Lemma}
\newtheorem{corollary}[theorem]{Corollary}
\theoremstyle{definition}
\newtheorem{definition}[theorem]{Definition}
\newtheorem{assumption}[theorem]{Assumption}
\theoremstyle{remark}
\newtheorem{remark}[theorem]{Remark}

% Todonotes is useful during development; simply uncomment the next line
%    and comment out the line below the next line to turn off comments
%\usepackage[disable,textsize=tiny]{todonotes}
\usepackage[textsize=tiny]{todonotes}


% The \icmltitle you define below is probably too long as a header.
% Therefore, a short form for the running title is supplied here:
\icmltitlerunning{Project Report Template for Data Literacy 2023/24}

\begin{document}

\twocolumn[
\icmltitle{My Data Literacy Project\\ (Replace this with your Project Title)}

% It is OKAY to include author information, even for blind
% submissions: the style file will automatically remove it for you
% unless you've provided the [accepted] option to the icml2023
% package.

% List of affiliations: The first argument should be a (short)
% identifier you will use later to specify author affiliations
% Academic affiliations should list Department, University, City, Region, Country
% Industry affiliations should list Company, City, Region, Country

% You can specify symbols, otherwise they are numbered in order.
% Ideally, you should not use this facility. Affiliations will be numbered
% in order of appearance and this is the preferred way.
\icmlsetsymbol{equal}{*}

\begin{icmlauthorlist}
\icmlauthor{Abdullah Abdul-Latif}{equal,first}
\icmlauthor{Lisa-Maria Fritsch}{equal,second}
\icmlauthor{Paul Kaifler}{equal,third}
\icmlauthor{Maximilian Schnitt}{equal,fourth}
\end{icmlauthorlist}

% fill in your matrikelnummer, email address, degree, for each group member
\icmlaffiliation{first}{Matrikelnummer 12345678, first.last@student.uni-tuebingen.de, MSc Machine Learning}
\icmlaffiliation{second}{Matrikelnummer 12345678, first.last@student.uni-tuebingen.de, MSc Computer Science}
\icmlaffiliation{third}{Matrikelnummer 12345678, first.last@student.uni-tuebingen.de, MSc Media Informatics}
\icmlaffiliation{fourth}{Matrikelnummer 6040570, maximilian.schnitt@student.uni-tuebingen.de, BSc Computer Science}

% You may provide any keywords that you
% find helpful for describing your paper; these are used to populate
% the "keywords" metadata in the PDF but will not be shown in the document
\icmlkeywords{Machine Learning, ICML}

\vskip 0.3in
]

% this must go after the closing bracket ] following \twocolumn[ ...

% This command actually creates the footnote in the first column
% listing the affiliations and the copyright notice.
% The command takes one argument, which is text to display at the start of the footnote.
% The \icmlEqualContribution command is standard text for equal contribution.
% Remove it (just {}) if you do not need this facility.

%\printAffiliationsAndNotice{}  % leave blank if no need to mention equal contribution
\printAffiliationsAndNotice{\icmlEqualContribution} % otherwise use the standard text.

%Put your abstract here. Abstracts typically start with a sentence motivating why the subject is interesting. Then mention the data, methodology or methods you are working with, and describe results. 
\begin{abstract}
As often mentioned in the news and in the daily life, people suggest that a higher 
\end{abstract}
%Motivate the problem, situation or topic you decided to work on. 
% Describe why it matters (is it of societal, economic, scientific value?). 
% Outline the rest of the paper (use references, e.g.~to \Cref{sec:methods}: 
% What kind of data you are working with, how you analyse it, and what kind of conclusion you reached. 
% The point of the introduction is to make the reader want to read the rest of the paper.

\section{Introduction}\label{sec:intro}
As often mentioned in the news and in the daily life, people suggest that a higher expected salary influences the job aspirations of the youth. 
We are curious about the underlying truth of thgis stetement. We have three Datasets, one Datset shows the Expected Salaries for various industrial sectors,
the second one shows the amount of students in the STEM sector in Tübingen from the Wintersemester 2005/2006 to the Wintersemester 2023/2024. 
The third dataset contains information about the amount of students, who finish high school in the certain year.
We did a feature analysis of the three datasets to get a better understanding of the data but the main focus of our analysis is on predicting 
the amount of students in the STEM sector for the next years. We used different approaches like a Random Forest approach with SARIMA
and a Multivariate Autoregression Model to investigate which approach works best for our purpose.


% In this section, describe \emph{what you did}. Roughly speaking, 
% explain what data you worked with, how or from where it was collected, it's structure and size.
%  Explain your analysis, and any specific choices you made in it. Depending on the nature of your project,
% you may focus more or less on certain aspects. If you collected data yourself, explain the collection process in detail.
% If you downloaded data from the net, show an exploratory analysis that builds
% intuition for the data, and shows that you know the data well. 
% If you are doing a custom analysis, explain how it works and why it is the right choice.
% If you are using a standard tool, it may still help to briefly outline it. Cite relevant works. 
% You can use the \verb|\citep| and \verb|\citet| commands for this purpose \citep{mackay2003information}.
\section{Data and Methods}\label{sec:methods}
We used three different pre collected datasets, our University data was collected by the Statistics Department from the University of Tuebingen. 
They used the immatriculation data from the students to create this dataset, which leads to
a quite accurate data which is necessary for the usage in our models.
\textcolor{red}{// HERE WE NEED MORE INFORMATION ABOUT THE DATA FROM THE OTHER DATASETS//}\\
The first step was to do some basic feature engineering on the datasets. We used various plots like line graphs, bar charts and pie charts
to get a better understanding of the data and detect trends. 
After we finished the basic feauture engineering we started the main analysis of the Vector Autoregressive Model (VAR). 
To get the most of this model, we wrote a parser, which parses text data generated from a PDF. This parser basically uses regular expressions to extract the amount of students
per years divided into the semesters and created a new dataset with the usage of pandas.\\
We've chosen the vector autoregressive model, because we have multivariate data and want to get a deeper understanding, if one of the datasets affects the others, 
so that we are able to predict one time series with another. A vector autoregressive model doesn't predict a single time series analyses, 
but multiple. Each time series is a linear combination which consists of n lags and an error term. A lag is the data from the past which we use to estimate
a new data point. We define the amount of lags in the hyperparameter p which defines the maxamount of lags which are used in the linear combination.
This linear combination also uses an error term, because we can't guarantee that our model predicts the correct output.\\
\textcolor{red}{One crucial step in the usage of those AR models, is that we have to ensure that our data is stationary. 
Data is stationary if those three attributes are fulfilled
\begin{enumerate}
    \item The mean $\mu$ is constant
    \item The standard deviation is constant
    \item There is no seasonality in the data
\end{enumerate}
// CHECK IF OUR DATA FULFILLS THOSE THREE ATTRIBUTES ///\\
After we ensured this, we did a PCAF to find statistically significant lags which we could use to find an optimal hyperparameter p of our vector autoregressive model.
A PCAF only represents the direct influence of the past datapoint $S_{n-i}$ for $i \in \{0,1,...,n\}$  to $S_n$. This gives us a graphical visualization with an error band of non statistical significant
lags in // GIVE A RANGE //.\\
After we used this PCAF we tried the different amount of lags, which looked like they have a significant input on the current datapoint $S_n$.
With the optimal lag we did a Granger Causality Test to get an understanding if one timeseries really influences another. The Granger Causality Test, does a t-Test and F-Test
checks if we can predict $S_i$ with $i \in \{2,3\}$ the time series $S_j$ $i\neq j$. The Granger Causality Test internally injects lags from the time series $S_J$ 
into the linear combination of $S_i$, then it does a t-Test and an F-test. As soon as one of those terms is verified by both tests,
then the test result is true.
}





% This is the template for a figure from the original ICML submission pack. In lecture 10 we will discuss plotting in detail.
% Refer to this lecture on how to include figures in this text.
% 
% \begin{figure}[ht]
% \vskip 0.2in
% \begin{center}
% \centerline{\includegraphics[width=\columnwidth]{icml_numpapers}}
% \caption{Historical locations and number of accepted papers for International
% Machine Learning Conferences (ICML 1993 -- ICML 2008) and International
% Workshops on Machine Learning (ML 1988 -- ML 1992). At the time this figure was
% produced, the number of accepted papers for ICML 2008 was unknown and instead
% estimated.}
% \label{icml-historical}
% \end{center}
% \vskip -0.2in
% \end{figure}

\section{Results}\label{sec:results}

In this section outline your results. At this point, you are just stating the outcome of your analysis. 
You can highlight important aspects (``we observe a significantly higher value of $x$ over $y$''), but leave interpretation and opinion to the next section. This section absoultely \emph{has} to include at least two figures.

\section{Discussion \& Conclusion}\label{sec:conclusion}

Use this section to briefly summarize the entire text. Highlight limitations and problems, but also make clear statements where they are possible and supported by the analysis. 

\section*{Contribution Statement}

Explain here, in one sentence per person, what each group member contributed. For example, you could write: Max Mustermann collected and prepared data. Gabi Musterfrau and John Doe performed the data analysis. Jane Doe produced visualizations. All authors will jointly wrote the text of the report. Note that you, as a group, a collectively responsible for the report. Your contributions should be roughly equal in amount and difficulty.

\section*{Notes} 

Your entire report has a \textbf{hard page limit of 4 pages} excluding references. (I.e. any pages beyond page 4 must only contain references). Appendices are \emph{not} possible. But you can put additional material, like interactive visualizations or videos, on a githunb repo (use \href{https://github.com/pnkraemer/tueplots}{links} in your pdf to refer to them). Each report has to contain \textbf{at least three plots or visualizations}, and \textbf{cite at least two references}. More details about how to prepare the report, inclucing how to produce plots, cite correctly, and how to ideally structure your github repo, will be discussed in the lecture, where a rubric for the evaluation will also be provided.


\bibliography{bibliography}
\bibliographystyle{icml2023}

\end{document}


% This document was modified from the file originally made available by
% Pat Langley and Andrea Danyluk for ICML-2K. This version was created
% by Iain Murray in 2018, and modified by Alexandre Bouchard in
% 2019 and 2021 and by Csaba Szepesvari, Gang Niu and Sivan Sabato in 2022.
% Modified again in 2023 by Sivan Sabato and Jonathan Scarlett.
% Previous contributors include Dan Roy, Lise Getoor and Tobias
% Scheffer, which was slightly modified from the 2010 version by
% Thorsten Joachims & Johannes Fuernkranz, slightly modified from the
% 2009 version by Kiri Wagstaff and Sam Roweis's 2008 version, which is
% slightly modified from Prasad Tadepalli's 2007 version which is a
% lightly changed version of the previous year's version by Andrew
% Moore, which was in turn edited from those of Kristian Kersting and
% Codrina Lauth. Alex Smola contributed to the algorithmic style files.
